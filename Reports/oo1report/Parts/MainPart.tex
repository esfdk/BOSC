\chapter{Opgavebesvarelse}
\section{Forord}

\section{Funktionalitet}
Krav til funktionalitet:
\begin{my_enumerate}
\item bosh skal køre uafhængigt.
\item Display af hostname.
\item Bruger skal kunne kalde simple kommandoer. Udskriv "Command not found" meddelelse, hvis kommando ikke findes.
\item Kommandoer skal kunne køres som baggrundsprocesser.
\item Det skal være mulighed for at lave redirect af stdin og stdout.
\item Mulighed for at anvende pipes.
\item Funktionen exit skal være indbygget.
\item Cntrl+C skal afslutte det program, som kører, men ikke bosh shell'en.
\end{my_enumerate}

Ekstra funktionalitet, vi har valgt at lave:
\begin{my_enumerate}
\item Display af "current working directory".
\item Mulighed for at kalde "cd" for at skifte directory.
\end{my_enumerate}

\section{Beskrivelse af implementation}
\subsection{Delopgave 1}
Vi kalder ikke system() i vores kode. De mest relevante systemkald, vi laver, er pipe(), dup(), fork(), waitpid() og execvp(). Desuden har vi fileno(), close() og fopen() kald.
\subsubsection{Tests}

\subsection{Delopgave 2}
Vi ændrede 'gethostname' i bosh.c til 'get\_user\_and\_hostname'. Metoden tager en char pointer og sætter den til at pege på en string med formatet "user@hostname". Vi tager 'user' via getenv og vi finder hostname i "/proc/sys/kernel/hostname".
\\Se linje ?? i bosh.c.
\subsubsection{Tests}

\subsection{Delopgave 3}
Vi bruger execvp() til at køre programmer i vores shell. execvp() leder efter et program i \$PATH. Hvis den finder det, bliver programmet kørt og resten af det program, som execvp blev kørt fra, bliver termineret. 
\\Se linje ?? i bosh.c.

\subsubsection{Tests}

\subsection{Delopgave 4}
Hvis et program bliver kørt som en baggrundprocess, bliver processen tilføjet til vores array af baggrundsprocessid'er. Desuden kalder vi ikke 'waitpid()' på det processid. 
\\Se linje ?? i bosh.c.
\\Vi undgår zombieprocesser ved at kalde 'signal(SIGCHLD, SIG\_IGN)'.
\subsubsection{Tests}

\subsection{Delopgave 5}
Hvis der i shellcmd bliver redirected stdin, stdout og/eller stderr, så kalder vi close() på dem, som bliver redirected. Derefter åbner vi filen, som er blevet redirected til. Vi kalder så dup() på den fil.
\\Se fx linje ?? i bosh.c.
\subsubsection{Tests}

\subsection{Delopgave 6}
Hvis der er mere end en enkelt command i 'shellcmd', når vi modtager den som argumentet til funktionen 'shell\_cmd\_with\_pipes', piper vi vores fd. (Linje ?? i bosh.c).
\\Vi kalder så close() på write delen af vores pipe (linje ??) og dup() read. Hvis 'write\_pipe' er mere end 0, lukker vi stdout og kalder dup på vores 'write\_pipe'.
\\Hvis vores 'write\_pipe' er 0 udenfor vores fork, så kalder vi close() på 'write\_pipe'. Hvis der er flere kommandoer tilbage i vores 'shellcmd', kalder vi 'shell\_cmd\_with\_pipes' med fd[1] (write delen af vores pipe) som parameter 'write\_pipe'.
\subsubsection{Tests}

\subsection{Delopgave 7}
'executeshellcmd' checker om den sidste kommando er "exit". Hvis den er, kalder vi 'exit(0)' i shellen. 
\\Se linje ?? i bosh.c.
\subsubsection{Tests}

\subsection{Delopgave 8}
I begyndelsen af vores main-metode, kalder vi 'signal(SIGINT, int\_handler)'. Dette betyder, at 'int\_handler' bliver kørt, hver gang bosh programmet modtager en interupt. 'int\_handler' er vores interupt handler, som lukker alle forgrundsprocesser. 
\\Se linje ?? i bosh.c.
\subsubsection{Tests}


\subsection{Ekstra funktionalitet 1}
\subsubsection{Tests}
\subsection{Ekstra funktionalitet 2}
\subsubsection{Tests}