\chapter{Opgavebesvarelse}
\section{Forord}

\section{Funktionalitet}
Krav til funktionalitet:
\begin{my_enumerate}
\item bosh skal køre uafhængigt.
\item Display af hostname.
\item Bruger skal kunne kalde simple kommandoer. Udskriv "Command not found" meddelelse, hvis kommando ikke findes.
\item Kommandoer skal kunne køres som baggrundsprocesser.
\item Det skal være mulighed for at lave redirect af stdin og stdout.
\item Mulighed for at anvende pipes.
\item Funktionen exit skal være indbygget.
\item Cntrl+C skal afslutte det program, som kører, men ikke bosh shell'en.
\end{my_enumerate}

Ekstra funktionalitet, vi har valgt at lave:
\begin{my_enumerate}
\item Display af "current working directory".
\item Mulighed for at kalde "cd" for at skifte directory.
\end{my_enumerate}

\section{Beskrivelse af implementation}
\subsection{Delopgave 1: Køre uafhængigt}
Vi kalder ikke system() i vores kode. De mest relevante systemkald, vi laver, er pipe(), dup(), fork(), waitpid() og execvp(). Desuden har vi fileno(), close() og fopen() kald.
\subsubsection{Tests}
Vi har ikke lavet nogen decideret test af dette, 
\subsection{Delopgave 2: Display hostname}
Vi ændrede 'gethostname' i bosh.c til 'get\_user\_and\_hostname'. Metoden tager en char pointer og sætter den til at pege på en string med formatet "user@hostname". Vi tager 'user' via getenv og vi finder hostname i "/proc/sys/kernel/hostname".
\\Se linje 27 - 43 same 222-227 i bosh.c.
\subsubsection{Tests}

\subsection{Delopgave 3: Kør programmer}
Vi bruger execvp() til at køre programmer i vores shell. execvp() leder efter et program i \$PATH. Hvis den finder det, bliver programmet kørt og resten af det program, som execvp blev kørt fra, bliver termineret. 
\\Se linje 141 i bosh.c.

\subsubsection{Tests}

\subsection{Delopgave 4: Baggrundsprocess}
Hvis et program bliver kørt som en baggrundprocess, bliver processen tilføjet til vores array af baggrundsprocessid'er. Desuden kalder vi ikke 'waitpid()' på det processid. 
\\Se linje 172-175 samt 193-204 i bosh.c.
\\Vi undgår zombieprocesser ved at kalde 'signal(SIGCHLD, SIG\_IGN)' (linje 213).
\subsubsection{Tests}

\subsection{Delopgave 5: Redirect af stdin/out}
Hvis der i shellcmd bliver redirected stdin, stdout og/eller stderr, så kalder vi close() på dem, som bliver redirected. Derefter åbner vi filen, som er blevet redirected til. Vi kalder så dup() på den fil.
\\Se fx linje 118 i bosh.c.
\subsubsection{Tests}

\subsection{Delopgave 6: Pipes}
Hvis der er mere end en enkelt command i 'shellcmd', når vi modtager den som argumentet til funktionen 'shell\_cmd\_with\_pipes', piper vi vores fd. (Linje 97 i bosh.c).
\\Vi kalder så close() på write delen af vores pipe (linje 105). Hvis der er flere commands i shellcmd, lukker vi stdin og dup() read delen af pipe. Hvis 'write\_pipe' er mere end 0, lukker vi stdout og kalder dup på vores 'write\_pipe'.
\\Hvis vores 'write\_pipe' er 0 udenfor vores fork, så kalder vi close() på 'write\_pipe'. Hvis der er flere kommandoer tilbage i vores 'shellcmd', kalder vi 'shell\_cmd\_with\_pipes' med fd[1] (write delen af vores pipe) som parameter 'write\_pipe'.
\subsubsection{Tests}

\subsection{Delopgave 7: Exit}
'executeshellcmd' checker om den sidste kommando er "exit". Hvis den er, returner vi 1 til main-funktionen, som så afslutter.
\\Se linje 58 i bosh.c.
\subsubsection{Tests}

\subsection{Delopgave 8: Ctrl+C}
I begyndelsen af vores main-metode, kalder vi 'signal(SIGINT, int\_handler)'. Dette betyder, at 'int\_handler' bliver kørt, hver gang bosh programmet modtager en interupt. 'int\_handler' er vores interupt handler, som lukker alle forgrundsprocesser. 
\\Se linje 210 samt 252-263 i bosh.c.
\subsubsection{Tests}


\subsection{Ekstra funktionalitet 1: Display current working directory}
Vi bruger en funktion 'getcwd' til at finde current working directory. Dette har vi implementeret i 'getcurrentdir'.
\\Se linje (47-53) i bosh.c.
\subsubsection{Tests}

\subsection{Ekstra funktionalitet 2: Change directory}
I vores 'executeshellcmd' tjekker vi om kommandoen er "cd" om den har mere end et argument. I så fald så kalder vi chdir på det første argument til "cd" kommandoen. Hvis det giver en error, printer vi det og returner 0.
\\Se linje 62-79 i bosh.c.
\subsubsection{Tests}
