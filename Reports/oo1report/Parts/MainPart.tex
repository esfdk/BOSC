\chapter{Opgavebesvarelse}
\section{Forord}
I denne rapport dokumenterer vi vores valg i forhold til implementationen af BOSC shell (bosh). 
\\Vores implementation kan findes i "Source Code/bosh.c". Vi har bygget videre på den implementation, som blev lagt op på bloggen\footnote{https://blog.itu.dk/BOSC-E2012/files/2012/09/oo1.zip}. Vi har dog erstattet "parser.c" med v2 af  "parser.c"\footnote{ https://blog.itu.dk/BOSC-E2012/files/2012/09/parserc.zip}.

Kildekode og testdokumentation kan findes i appendix på side \pageref{Appendix}. Vores git repository kan findes på https://github.com/esfdk/BOSC .
Når vi ser tilbage på processen, føler vi, at der er et par ting, vi kunne have gjort anderledes.
\begin{my_itemize}
\item Gjort det muligt at få en liste af baggrundsprocesser. (fx via en "jobs" kommando)
\item Gjort det muligt at hente baggrundsprocesser frem til forgrund.
\item Delt implementationen ud i flere dele (istedet for en lang bosh.c fil)
\end{my_itemize}

Vi har fået en del hjælp til pipes, redirect af stdin/out og "cd" kommandoen af Ulrik Damm (ulfd@itu.dk). Hans GitHub repository er https://github.com/ulrikdamm/progsh .

Der er en "bug" i parseren, som gør, at den ikke er i stand til at læse argumenter med mellemrum. Fx vil kommandoen 'mkdir "mappe med mellemrum" ' ikke lave en mappe ved navn "mappe med mellemrum", men derimod lave tre mapper med navnene '"mappe', 'med' 'mellemrum"'.

\section{Funktionalitet}
Krav til funktionalitet:
\begin{my_enumerate}
\item bosh skal køre uafhængigt.
\item Display af hostname.
\item Bruger skal kunne kalde simple kommandoer. Udskriv "Command not found" meddelelse, hvis kommando ikke findes.
\item Kommandoer skal kunne køres som baggrundsprocesser.
\item Det skal være mulighed for at lave redirect af stdin og stdout.
\item Mulighed for at anvende pipes.
\item Funktionen exit skal være indbygget.
\item Cntrl+C skal afslutte det program, som kører, men ikke bosh shell'en.
\end{my_enumerate}

Ekstra funktionalitet, vi har valgt at lave:
\begin{my_enumerate}
\item Display af "current working directory".
\item Mulighed for at kalde "cd" for at skifte directory.
\end{my_enumerate}

\section{Beskrivelse af implementation}
\subsection{Delopgave 1: Køre uafhængigt}
\label{D1}
Vi kalder ikke system() i vores kode. De mest relevante systemkald, vi laver, er pipe(), dup(), fork(), waitpid() og execvp(). Desuden har vi fileno(), close() og fopen() kald.
\subsubsection{Tests}
Vi har ikke lavet nogen decideret test af dette.

\subsection{Delopgave 2: Display hostname}
\label{D2}
Vi har valgt at vise både user og hostname i shell'en, da vi havde ønske om, at vores shell skulle ligne Bash shell så meget som muligt.
\\Vi ændrede 'gethostname' i bosh.c til 'get\_user\_and\_hostname'. Metoden tager en char pointer og sætter den til at pege på en string med formatet "user@hostname". Vi tager 'user' via getenv og vi finder hostname i "/proc/sys/kernel/hostname".
\\Se linje 27 - 43 same 222-227 i bosh.c.
\subsubsection{Tests}
Vi tjekker om vores shell viser det samme navn, som terminalen gør.
\\Forventet resultat: Se billede \ref{Test2} på side \pageref{Test2}.
\\Faktisk resultat: Se billede \ref{Test2} på side \pageref{Test2}.

\subsection{Delopgave 3: Kør programmer}
\label{D3}
Vi bruger execvp() til at køre programmer i vores shell. execvp() leder efter et program i \$PATH. Hvis den finder det, bliver programmet kørt, og resten af det program, som execvp blev kørt fra, bliver termineret. 
\\Se linje 141 i bosh.c.
\subsubsection{Tests}
\paragraph{Test3.1}
Vi kører kommandoen ls i den normale shell, og derefter i vores egen shell til sammmenligning.
\\Forventet resultat: Se billede \ref{Test3_1} på side \pageref{Test3_1}.
\\Faktisk resultat: Se billede \ref{Test3_1} på side \pageref{Test3_1}.

\paragraph{Test3.2}
Vi kører kommandoen cat i den normale shell, og derefter i vores egen shell til sammenligning.
\\Forventet resultat: Se billede \ref{Test3_2} på side \pageref{Test3_2}.
\\Faktisk resultat: Se billede \ref{Test3_2} på side \pageref{Test3_2}.

\paragraph{Test3.3}
Vi kører kommandoen wc i den normale shell, og derefter i vores egen shell til sammenligning.
\\Forventet resultat: Se billede \ref{Test3_3} på side \pageref{Test3_3}.
\\Faktisk resultat: Se billede \ref{Test3_3} på side \pageref{Test3_3}.

\paragraph{Test3.4}
Vi kører kommandoen "42" i den normale shell, og derefter i vores egen shell til sammenligning.
\\Forventet resultat: Se billede \ref{Test3_4} på side \pageref{Test3_4}.
\\Faktisk resultat: Se billede \ref{Test3_4} på side \pageref{Test3_4}.

\subsection{Delopgave 4: Baggrundsprocess}
\label{D4}
Hvis et program bliver kørt som en baggrundprocess, bliver processen tilføjet til vores array af baggrundsprocessid'er. Desuden kalder vi ikke 'waitpid()' på det processid. 
\\Se linje 172-175 samt 193-204 i bosh.c.
\\Vi undgår zombieprocesser ved at kalde 'signal(SIGCHLD, SIG\_IGN)' (linje 213).
\subsubsection{Tests}
Vi kører 'sleep 100 \&' og derefter kommandoen date for at vise at sleep kører i baggrunden.
\\Forventet resultat: Forskellen mellem de to tidspunkter fra 'date" kommandoer er mindre end 100 sekunder.
\\Faktisk resultat: Se billede \ref{Test4} på side \pageref{Test4}.

\subsection{Delopgave 5: Redirect af stdin/out}
\label{D5}
Hvis der i shellcmd bliver redirected stdin, stdout og/eller stderr, så kalder vi close() på dem, som bliver redirected. Derefter åbner vi filen, som er blevet redirected til. Vi kalder så dup() på den fil.
\\Se fx linje 118 i bosh.c.
\subsubsection{Tests}
Vi kører kommandoen 'wc -l \textless /etc/passwd \textgreater antalkontoer' i den normale shell, og derefter i vores egen shell til sammenligning.
\\Forventet resultat: Se billede \ref{Test5} på side \pageref{Test5}.
\\Faktisk resultat: Se billede \ref{Test5} på side \pageref{Test5}.
 
\subsection{Delopgave 6: Pipes}
\label{D6}
Hvis der er mere end en enkelt command i 'shellcmd', når vi modtager den som argumentet til funktionen 'shell\_cmd\_with\_pipes', piper vi vores fd. (Linje 97 i bosh.c).
\\Vi kalder så close() på write delen af vores pipe (linje 105). Hvis der er flere commands i shellcmd, lukker vi stdin og dup() read delen af pipe. Hvis 'write\_pipe' er mere end 0, lukker vi stdout og kalder dup på vores 'write\_pipe'.
\\Hvis vores 'write\_pipe' er 0 udenfor vores fork, så kalder vi close() på 'write\_pipe'. Hvis der er flere kommandoer tilbage i vores 'shellcmd', kalder vi 'shell\_cmd\_with\_pipes' med fd[1] (write delen af vores pipe) som parameter 'write\_pipe'.
\subsubsection{Tests}
Test: Vi starter med at køre kommandoen ls i shellen, og derefter prøver vi at køre ls | wc -w
\\Forventet resultat: Se billede \ref{Test6} på side \pageref{Test6}.
\\Faktisk resultat: Se billede \ref{Test6} på side \pageref{Test6}.

\subsection{Delopgave 7: Exit}
\label{D7}
Første iteration af denne delopgave terminerede bare bosh (med exit(0)), men dette ændrede vi, da vi følte det gav bedre mening, hvis vi lod main-funktionen afslutte.
\\'executeshellcmd' checker om den sidste kommando er "exit". Hvis den er, returner vi 1 til main-funktionen, som så afslutter.
\\Se linje 58 i bosh.c.
\subsubsection{Tests}
Vi starter shellen og kører exit kommandoen.
\\Forventet resultat: Shellen afsluttes.
\\Faktisk resultat: Se billede \ref{Test7} på side \pageref{Test7}.

\subsection{Delopgave 8: Ctrl+C}
\label{D8}
Vores første implementation af Ctrl+C funktionaliteten stoppede alle processer (både forgrunds- og baggrundsprocesser). Dette lavede vi dog om, så det matchede måden, som Bash håndterer Cntrl+C signals. Vi har ikke implementeret en mulighed for at hente baggrundsprocesser frem i forgrunden, hvilket gør, at vi ikke kan afslutte baggrundsprocesser, medmindre de selv terminerer. Sådan funktionalitet bør være højt på dagsordenen, hvis man udvider funktionaliteten af BoSh.
\\I begyndelsen af vores main-metode, kalder vi 'signal(SIGINT, int\_handler)'. Dette betyder, at 'int\_handler' bliver kørt, hver gang bosh programmet modtager en interupt. 'int\_handler' er vores interupt handler, som lukker alle forgrundsprocesser. 
\\Se linje 210 samt 252-263 i bosh.c.
\subsubsection{Tests}
Vi kører kommandoen wc uden parameter og trykker ctrl-c for at se om processen afsluttes.
\\Forventet resultat: wc stoppes efter ctrl-c er blevet tastet.
\\Faktisk resultat: Se billede \ref{Test8} på side \pageref{Test8}.

\subsection{Ekstra funktionalitet 1: Display current working directory}
\label{E1}
Som nævnt i \textbf{\ref{D2}} har vi haft et ønske om at få vores shell til at ligne Bash, så derfor ville vi gøre det muligt at både se og skifte directory.
\\Vi bruger en funktion 'getcwd' til at finde current working directory. Dette har vi implementeret i 'getcurrentdir'.
\\Se linje (47-53) i bosh.c.
\subsubsection{Tests}
Vi starter shellen for at se om der vises det directory der i øjeblikket arbejdes i.
\\Forventet resultat: Se billede \ref{Test9} på side \pageref{Test9}.
\\Faktisk resultat: Se billede \ref{Test9} på side \pageref{Test9}.

\subsection{Ekstra funktionalitet 2: Change directory}
\label{E2}
Som nævnt i \textbf{\ref{E1}}, så ville vi gøre det muligt at både se og skifte directory, hvilket var grundlaget for at implementere dette.
\\I vores 'executeshellcmd' tjekker vi om kommandoen er "cd" om den har mere end et argument. I så fald så kalder vi chdir på det første argument til "cd" kommandoen. Hvis det giver en error, printer vi det og returner 0.
\\Se linje 62-79 i bosh.c.
\subsubsection{Tests}
Vi går ind i vores shell og prøver derefter at skifte til andet directory end det der arbejdes i.
\\Forventet resultat: Se billede \ref{Test10} på side \pageref{Test10}.
\\Faktisk resultat: Se billede \ref{Test10} på side \pageref{Test10}.