\chapter{Forord}
I denne rapport dokumenterer vi vores valg i forhold til implementationen af opgaverne i Obligatorisk opgave 3.

Kildekode og testdokumentation kan findes i appendix på side \pageref{Appendix}. Vores git repository kan findes på https://github.com/esfdk/BOSC/tree/master/oo3.

\chapter{Beskrivelse af implementation}
\section{Opgave 1}
\label{O1}

\subsection{Del 1}
\label{O1_1}
\begin{my_itemize}
\item[ADD] untagger de to øverste elementer på stakken, ligger dem sammen, tagger den nye værdi og ligger den på toppen af stakken.
\item[CSTI I] tager den næste værdi i p[ ] arrayet og ligger den på toppen af stakken.
\item[NIL] Ligger 0 på toppen af stakken. Hvis der kun ligger rent 0 bits betyder det NIL og ikke integer 0.
\item[IFZERO] tager det øverste element af stakken og decrementere stackpointeren med en. Den tjekker om v er en int. Hvis v er en int udtagges v og sammenlignes med nul , ellers sammenlignes v med NIL. Hvis sammenligning er sand i tilfældet med nul bliver program counter sat til den nuværende værdi på (......) ellers bliver næste instruktion udført.
\item[CONS] laver en cons celle ud af de to øverste elementer på stakken, og decrementere så stack pointeren med en.
\item[CAR] henter et word fra stakken og tjekker om det er NIL, hvis det ikke er NIL tages det første element af cons cellen og ligges på toppen af stakken i stedet for det hentede word.
\item[SETCAR] henter det øverste element på stakken og et word. Den tager wordets første værdi til at være den udhentede værdi på stakken.
\end{my_itemize}

\subsection{Del 2}
\label{O1_2}
\begin{my_itemize}
\item[Length] laver to right shifts, hvilket fjerner de to garbage collection bites. Derefter bruger den bitwise AND til at sammenligne length bitsne med 0x003FFFF. Dette giver os værdien af n bitsne.
\item[Color] går ind og bruger bitwise AND til at sammenligne farven på cellen med 0011 hvorved den finder cellens faktiske farve.
\item[Paint] går ind og ændrer gg til den color der er blevet givet med som argument. eksempelvis laver Paint med arguemnt BLUE gg om til 11.
\end{my_itemize}

\subsection{Del 3}
\label{O1_3}
allocate( ) bliver kun kaldt i CONS casen.  (umiddelbart ikke andre stedder ind i interpretation loopet)

\subsection{Del 4}
\label{O1_4}
Når der bliver allokeret og der ikke er noget free space.

\section{Opgave 2}
\label{O2}

\section{Opgave 3}
\label{O3}

\section{Opgave 4}
\label{O4}

\section{Opgave 5}
\label{O5}

\section{Opgave 6}
\label{O6}
